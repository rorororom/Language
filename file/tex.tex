\documentclass{article}
\usepackage[a4paper]{geometry}
\geometry{top=1.41cm, bottom=1.41cm, left=1.41cm, right=1.41cm, marginparwidth=1.75cm}
\usepackage[T2A]{fontenc}
\usepackage[utf8]{inputenc}
\usepackage[english, russian]{babel}
\usepackage{amsmath}
\usepackage{graphicx}
\usepackage[colorlinks=true, allcolors=blue]{hyperref}
\usepackage{amsfonts}
\usepackage{amssymb}
\usepackage{seqsplit}
\usepackage[dvipsnames]{xcolor}
\usepackage{enumitem}
\usepackage{algorithm}
\usepackage{algpseudocode}
\usepackage{algorithmicx}
\usepackage{mathalfa}
\usepackage{mathrsfs}
\usepackage{dsfont}
\usepackage{caption,subcaption}
\usepackage{wrapfig}
\usepackage[stable]{footmisc}
\usepackage{indentfirst}
\usepackage{rotating}
\usepackage{pdflscape}
\usepackage{MnSymbol,wasysym}
\title{<<To differentiate or not to differentiate? That is the question.>>}
\author{Ромашкина Мария Б05-332}
\begin{document}
\maketitle
\newpage
We believe that in a world full of changes and surprises, the ability to distinguish,                   analyze, and highlight key moments is a true art. Differentiation is like a reliable map in the labyrinth of life,                   helping us navigate the turbulent reality. Imagine this: you stand on top of a mountain,                   and your knowledge of differential equations is like wings that allow you to soar freely in the air of change.                   It's not just mathematics; it's the ability to see beauty in structures and relationships. And if someone ever asks you,                   to differentiate or not to differentiate, you will definitely know the answer -                   let's unleash our potential and highlight the key moments in this exciting journey called Life.

But, for now, while you are young...                   Party and enjoy yourself because youth will never come back,                   but knowing how to take derivatives, you can always Google that)

If someone asks you why you're so tired:                   Differentiation doesn't take vacations. It just goes to infinity.
The table of derivatives we will be using:
$$sin(x)' =  cos( x ) \cdot 1 $$
$$cos(x)' =  -1 \cdot sin( x ) \cdot 1 $$
$$tg(x)' =  \frac { 1 } { ( cos( x ) ) ^ { 2 } } \cdot 1 $$
$$cot(x)' =  \frac { 1 } { ( sin( x ) ) ^ { 2 } } \cdot -1 $$
$$arcsin(x)' =  \frac { 1 } { \sqrt{ 1 - x ^ { 2 } } } $$
$$arccos(x)' =  \frac { -1 \cdot 1 } { \sqrt{ 1 - x ^ { 2 } } } $$
$$arctg(x)' =  \frac { 1 } { 1 + x ^ { 2 } } $$
$$arccot(x)' =  \frac { -1 \cdot 1 } { 1 + x ^ { 2 } } $$
$$ln(x)' =  \frac { 1 } { x } \cdot 1 $$

\includegraphics[height=10cm]{img/img1_59077.png}

$
\text{When you differentiate, imagine that your variables are your favorite dumplings, and you're separating them to discover their secret recipe.}$
$$f(x) =  sin( 15 \cdot x + 6 ) ^ { 3 } + cos( 15 \cdot x + 6 ) $$
$
\text{The key is to breathe calmly, then you can obtain:}$
$$f'(x) =  3 \cdot ( sin( 15 \cdot x + 6 ) ) ^ { 3 - 1 } \cdot ( 0 \cdot x + 15 \cdot 1 + 0 ) * 0 \cdot x + 15 \cdot 1 + 0 + ( 0 \cdot x + 15 \cdot 1 + 0 ) * 0 \cdot x + 15 \cdot 1 + 0 $$

\includegraphics[height=10cm]{img/img2_61298.png}

$
\text{When you differentiate, imagine that your variables are your favorite dumplings, and you're separating them to discover their secret recipe.}$
$$f'(x) =  3 \cdot ( sin( 15 \cdot x + 6 ) ) ^ { 2 } \cdot ( 0 \cdot x + 15 \cdot 1 + 0 ) * 0 \cdot x + 15 \cdot 1 + 0 + ( 0 \cdot x + 15 \cdot 1 + 0 ) * 0 \cdot x + 15 \cdot 1 + 0 $$
$
\text{This is trivial.}$
$$f'(x) =  3 \cdot ( sin( 15 \cdot x + 6 ) ) ^ { 2 } \cdot ( 0 \cdot x + 15 \cdot 1 ) * 0 \cdot x + 15 \cdot 1 + ( 0 \cdot x + 15 \cdot 1 + 0 ) * 0 \cdot x + 15 \cdot 1 + 0 $$
$
\text{Differentiate as if your life depends on it... and maybe it really does...}$
$$f'(x) =  3 \cdot ( sin( 15 \cdot x + 6 ) ) ^ { 2 } \cdot ( 0 + 15 \cdot 1 ) * 0 + 15 \cdot 1 + ( 0 \cdot x + 15 \cdot 1 + 0 ) * 0 \cdot x + 15 \cdot 1 + 0 $$
$
\text{It's so difficult to live when there are only 24 hours in a day, it would be easier if there were more!}$
$$f'(x) =  3 \cdot ( sin( 15 \cdot x + 6 ) ) ^ { 2 } \cdot ( 0 + 15 ) * 0 + 15 + ( 0 \cdot x + 15 \cdot 1 + 0 ) * 0 \cdot x + 15 \cdot 1 + 0 $$
$
\text{It is obvious that it turns out like this:}$
$$f'(x) =  3 \cdot ( sin( 15 \cdot x + 6 ) ) ^ { 2 } \cdot ( 0 + 15 ) * 0 + 15 + ( 0 \cdot x + 15 \cdot 1 ) * 0 \cdot x + 15 \cdot 1 $$
$
\text{It is obvious that it turns out like this:}$
$$f'(x) =  3 \cdot ( sin( 15 \cdot x + 6 ) ) ^ { 2 } \cdot ( 0 + 15 ) * 0 + 15 + ( 0 + 15 \cdot 1 ) * 0 + 15 \cdot 1 $$
$
\text{This is trivial.}$
$$f'(x) =  3 \cdot ( sin( 15 \cdot x + 6 ) ) ^ { 2 } \cdot ( 0 + 15 ) * 0 + 15 + ( 0 + 15 ) * 0 + 15 $$
$
\text{As they say, well, here it is:}$
$$f'(x) =  3 \cdot ( sin( 15 \cdot x + 6 ) ) ^ { 2 } \cdot cos( 15 \cdot x + 6 ) \cdot 15 + ( 0 + 15 ) * 0 + 15 $$
$
\text{When your equation is complex, differentiation brings ease into life!}$
$$f'(x) =  3 \cdot ( sin( 15 \cdot x + 6 ) ) ^ { 2 } \cdot cos( 15 \cdot x + 6 ) \cdot 15 + -1 \cdot sin( 15 \cdot x + 6 ) \cdot 15 $$
\newpage\begin{thebibliography}{9}\bibitem{lukashov}Lukashov, Alexey Leonidovich.\end{thebibliography}\end{document}
